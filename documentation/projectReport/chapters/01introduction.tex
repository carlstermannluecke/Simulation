\ifgerman{\chapter{Einleitung}}{\chapter{Introduction}}
%- Hintergrund
%- Motivation
%- Ziele
%- Aufgaben
%- Allgemeine Beschreibung des Projektes
%- Worum geht es in dieser Arbeit?
%- Wer hat die Arbeit veranlasst und wozu?
%- Wer soll von den Ergebnissen profitieren?
%- Welches Problem soll gelöst werden? Warum?
%- Unter welchen Umständen braucht man eine Verbesserung?
%- Was ist der Stand der Technik?
%- Welche noch offenen Probleme gibt es?
%- Worin unterscheidet sich mein Ansatz von den bisherigen?
%- Welche Ziele hat die Arbeit?
%- Wie will ich diese Ziele erreichen?
%- Was habe ich im Einzelnen vor?


\section{Motivation}

\todo{project context of this work}

\todo{who needs the results}

\todo{what are the problems to be solved?}

\todo{what are existing solutions, what's different in this approach, what is the improvement}


       


  
\section{Problem Statement}
    
    \todo{what are the goals}
    \todo{how are we going to reach this goals}
    \todo{what is to be done}
 
         
\section{Outline}
    
  \todo{short description of the sections}
   % use " in \ref{} to reference "\labels{}" in the document
   