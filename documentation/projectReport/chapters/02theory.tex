\chapter{Theory}
\label{sec:theo}

%In diesem Kapitel beschreiben Sie Ihren eigenen Beitrag
%- Es muss klar sein, worin die eigentliche Innovation besteht#


\section{Quadcopter Modelling}

\todo{fundamental physics}

\todo{particle simulation}



\section{Vrep}

\todo{connecting visual representation and physical model}

\todo{simulation structure (lua scripts, scene structure}

\todo{lua module structure}

\todo{external interface (signals)}

\section{Communication/Ivy-Bus}

test check in 
% example for definition    
% \begin{definitionnonum}[Softwarearchitektur ]
% Die Softwarearchitektur repräsentiert alle Softwarekomponenten und deren Interaktionen in einer hierarchischen Struktur. Es werden sowohl statische Aspekte wie Schnittstellen und Datenpfade zwischen Softwarekomponenten, als auch dynamische Aspekte wie Prozessabläufe und zeitliches Verhalten beschrieben.
% \end{definitionnonum} 
 

  
%  example for bulletpoints
%\begin{itemize}    
%	\item{(überarbeiteter) Sicherheitsplan nach ISO 26262-6:2011, 5.5.1}
%	\item{Design- und Programmierrichtlinien für Programmier- und Modellierungssprachen nach ISO 26262-6:2011, 5.5.5}
%	\item{Hardware-Software-Interfacespezifikation nach ISO 26262-4:2011, 7.5.6}
%	\item{Software-Sicherheitsanforderungen nach ISO 26262-6:2011, 6.5.1}
%	\item{(überarbeiteter) Software-Verifizierungsplan nach  ISO 26262-6:2011, 6.5.3}
%	\item{Software-Verifizierungsbericht nach ISO 26262-6:2011, 6.5.4}
%\end{itemize}

 
%example for table    
% \begin{table}[h]
%      \centering
%    \caption{7.4.1 Notationen für Softwarearchitekturen\cite{iso6}}
%    \begin{tabular}{|c|l|c|c|c|c|}
%      \hline
%     \multicolumn{2}{|c|}{\multirow{2}{*}{Methoden}} & \multicolumn{4}{|c|}{ASIL}\\
%      \multicolumn{2}{|c|}{} &A & B & C & D\\
%      \hline
%       1a & Informelle Notationen & ++ & ++ & + & +\\
%      \hline
%       1b & Semi-formale Notationen & + & ++ & ++ & ++\\
%      \hline
%        1c & Formale Notationen & + & + & + & +\\
% 
%      \hline
%      \end{tabular}
%      \label{tab:archDescr}
%\end{table}
