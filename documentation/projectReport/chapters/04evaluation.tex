\chapter{Evaluation}
\label{sec:eval}

\todo{how realisitc is the simulation?}
\todo{which properties can be modelled well, which can't?}
\section{Testing with Joystick}
 \begin{itemize}
\item{to initially test the vrep simulation and the JAVA communication, a setup was build to control the virtual copter with a gaming joystick}
\item{the JAVA programm was extended to send the commands for pitch, roll, yaw and thrust as the real copter would have sent them}
\item{this was to show that communication between JAVA and VREP via the ivy bus was possible and that the simulation was running stable enough for external control}
\item{flying with the joystick was really easy, only keeping the copter at the same height posed a litter bigger challenge. when detecting this probleme, the tuning of the throttle with the logistic curve was implemented, making it easier to hover the quadcopter}
\end{itemize}

\section{Speed}


\todo{communication delay}
\begin{itemize}

\item{mean delta t between sent messages, compare with the configured message frequency}
\item{ run this with two or multiple quadrocopter}
\item{latency of JAVA link}
\end{itemize}


\todo{vrep simulation speed}
\todo{run vrep on the 10core computer in the lab, look at mean execution time}
\todo{add multiple copter}


\section{Accuracy}
\todo{angular movement response }
\begin{itemize}
\item{plot graph of euler angles}
\item{highlight start of drift, other interesting elements?}
\end{itemize}
\todo{path of quadcopter}
\begin{itemize}

\item{using Medusa positioning system}
\item{shortly explain limitations/problems with the system}
\item{results are not perfect, but it is what is available}
\item{compare scatter plots of real and virtual quadcopter positions}
\end{itemize}

\begin{itemize}
\item{stability}
\item{time that both copter stay inside the arena}
\item{what happens when copter start to drift away?}
\item{maybe show angular movement plot at this time to find explanation there}
\end{itemize}


% example for including picture
%\begin{figure}
%\begin{center}
%\includegraphics[width=\textheight,angle=90]{SimulinkLibArchitect}
%\caption{Aufbau der Conqat Simulink Library}
%\label{slLibArch}
%\end{center}
%\end{figure}
%\begin{figure}
%\begin{center}
%\includegraphics[width=\textwidth]{archDoc}
%\caption{Aufbau des Software-Prototypen}
%\label{pic:saProto}
%\end{center}
%\end{figure}



% example for code listing
%\begin{lstlisting}[caption=Metric-Interface]
%interface Metric {
%/**
% * @return the  value of the metric as it is defined 
% * for the block/architecture and all its subblocks
% */
%    public double getGlobalValue();
%/**
% * @return the worst value of the metric among the 
% * block/architecture and all its subblocks
% */
%    public double getWorstValue();
%/**
% * @return the block with the worst value of the 
% * metric among the block/architecture and all its subblocks
% */
%    public ArchitectureBlock getMaxBlock();
%/**
% * @return a list of all values of the metric for the 
% * block/architecture and its subblocks
% */
%    public double[] getAllBlockValues();
%}
%\end{lstlisting}
%
%
%\begin{lstlisting}[caption=Metric-Interface, label=lst:filter]
%/**
%* disables all subBlocks that are masked in the simulink 
%* Model and have no input
%* @param currentBlock
%*/
%void maskedZeroInDisable(ArchitectureBlock currentBlock){
%   for (ArchitectureBlock subBlock :
%  currentBlock.getSubBlocks()){
%      if (subBlock.getInPorts().isEmpty() && 
%     subBlock.hasAttribute("MaskEnableString")){
%         subBlock.disable();
%      }
%      maskedZeroInDisable(subBlock);
%   }
%}
%\end{lstlisting}